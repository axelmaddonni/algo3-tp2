\subsection{Explicación formal del problema}

Sea una matriz $M$ $\in$ $\mathbb{R}^{n \times m}$, donde cada posición de la matriz $M_{ij}$ tiene un valor asociado $h_{ij}$. El problema consiste en calcular el camino mínimo de casilleros desde la posición (1,1) hasta la posición ($N$,$M$), donde los únicos dos  movimientos posibles son:

\emph{Moverse hacia el casillero superior}: \[M_{i,j} \rightarrow M_{i+1,j}\] 

ó \emph{Moverse hacia el casillero de la derecha}: \[M_{i,j} \rightarrow M_{i,j+1}\]

Realizar estos movimientos tiene un costo que depende de un parámetro de entrada $H$:

\begin{equation} \label{costoHaciaArriba}
   Costo(M_{i,j} \rightarrow M_{i+1,j}) = \left\{ 
     \begin{array}{lr}
       0 							& $ si $  |h_{i,j} - h_{i+1,j}| \leq H \\
       |h_{i,j} - h_{i+1,j}| - H 	&   $ caso contrario $ \\
     \end{array}
   \right.
\end{equation} 

\begin{equation}\label{costoHaciaDerecha}
  Costo(M_{i,j} \rightarrow M_{i,j+1}) = \left\{ 
     \begin{array}{lr}
       0 							& $ si $  |h_{i,j} - h_{i,j+1}| \leq H \\
       |h_{i,j} - h_{i,j+1}| - H 	&   $ caso contrario $ \\
     \end{array}
   \right.
\end{equation} 

%\subsubsection{Modelado del problema con grafos}
%
%Para resolver el problema vamos a modelar la matriz $M$ $\in$ $\mathbb{R}^{n \times m}$ con un grafo dirigido $G=(V,E)$, donde $V$ es el conjunto de nodos y $E$ el conjunto de aristas (orientadas) del grafo. Cada nodo $v_{i,j} \in V$ representa a una posición de la matriz $(i,j)$, y las aristas representarán los posibles movimientos entre dos nodos dentro de la matriz (\emph{Moverse hacia el casillero superior} y \emph{Moverse hacia el casillero de la derecha}). Es decir, 
%
%\begin{align*}
%(\forall i,i\prime \in [1..n],\ j,j\prime \in [1..m]) \ : & \\ 
%(v_{i,j}, v_{i\prime, j\prime}) \in E \hfill \ \ \ \Leftrightarrow \hfill \ \ \
%& (i\prime, j\prime) = (i+1,j) \\
%o\ & (i\prime, j\prime) = (i,j+1) \\
%\end{align*}
%
%Además, definimos $\kappa : E \rightarrow \mathbb{R}$ el peso de las aristas, como el costo que implica realizar dichos movimientos (dado por las funciones \ref{costoHaciaArriba} y \ref{costoHaciaDerecha}):
%\begin{equation}
%\kappa (v_{i,j}, v_{i\prime, j\prime}) = Costo (M_{i,j} \rightarrow M_{i\prime, j\prime})
%\end{equation}
% 
%Modelando el problema de esta manera, el problema se traduce en encontrar el camino mínimo (el camino de menor costo entre todas las aristas) en $G$ desde el vértice $v_{1,1}$ hasta el vértice $v_{n,m}$. 

\subsubsection{Ejemplos}

% Ejemplo de una matriz, y la solucion del camino. A TERMINAR

\subsection{Formulación Recursiva}

Obtendremos la solución del problema usando un algoritmo de programación dinámica. Para eso, planteamos una formulación recursiva del problema. Sea la función $f$:

$f(i,j) = $ costo de un camino óptimo desde $ M_{1,1} $ hasta $ M{i,j}$

Entonces, la solución del problema está dada por $f(n,m)$. 
Se propone la siguiente recursión para calcular $f$:

\begin{equation} \label{formulacionRecursiva}
f(i,j) = min \left( 
	\begin{array}{lr}
    f(i-1, j) + Costo(M_{i-1,j} \rightarrow M_{i,j}), \\
	f(i,j-1) + Costo(M_{i,j-1} \rightarrow M_{i,j}) \\
     \end{array}
   \right)
\end{equation}  

con algunos casos particulares:

$f(1,1) = 0 $

$f(1,j) = f(1,j-1) + Costo(M_{1,j-1} \rightarrow M_{1,j})$

$f(i,1) = f(i-1,1) + Costo(M_{i-1,1} \rightarrow M_{i,1})$

\subsubsection{Demostración de Correctitud}

El caso base $f(1,1)$ es trivial, ya que no realizo ningún movimiento.

Dada cualquier otra posición $(i,j)$ de la matriz, el camino mínimo para llegar a ella tiene como última posición visitada o la casilla de su izquierda $(i, j-1)$ (si existe), o la casilla de abajo $(i-1, j)$ (si existe) por el enunciado del problema, ya que los únicos movimientos posibles son moverse hacia arriba o hacia la derecha.

Para demostrar que la función $f$ propuesta calcula el camino mínimo hasta la casilla cualquiera $M_{i,j}$ debemos ver que se cumple el \textbf{Principio de Optimalidad.} Es decir, que dado un camino óptimo desde $M_{1,1}$ hasta $M{i,j}$, $P_{i,j}$, entonces, el subcamino desde $M_{1,1}$ hasta el inmediato antecesor de $M_{i,j}$ ($P_{i-1,j}$ o $P{i,j-1}$) debe ser óptimo: 

Sea $P_{i,j}$ el camino óptimo desde $M_{1,1}$ $hasta M{i,j}$.
SPGE, puedo suponer que el inmediato antecesor de $M_{i,j}$ en $P$ es $M_{i, j-1}$. Supongamos que elsub camino hasta el antecesor de $M_{i,j}$ ($P_{i,j-1}$)  no es óptimo. Entonces, $\exists$ $P\prime_{i,j-1}$ tal que $Costo(P\prime_{i,j-1})$ $<$ $Costo(P_{i,j-1})$.

Pero entonces, puedo tomar el camino  $P\prime_{i,j-1}$ y de ahí moverme a la posición $M_{i,j}$. 
$Costo (P\prime_{i,j-1}) + Costo(M_{i-1,j} \rightarrow M_{i,j}) < Costo (P{i,j-1}) + Costo(M_{i-1,j} \rightarrow M_{i,j}) = Costo(P_{i,j})$
Pero esto es absurdo, ya que existiría un camino mejor que el óptimo hasta la posición $(i,j)$.

Como aplica el principio de optimalidad y sólo hay dos posibles antecesores para una determinada casilla $M_{i,j}$, para calcular el camino mínimo hasta una posición $(i,j)$ basta con tomar el mínimo entre las dos posibilidades, tomando el cuenta el costo de realizar el último movimiento.

Para los casos borde de la matriz, en la primera columna y en la primera fila, donde sólo tengo un sólo posible antecesor (el inmediato de abajo y el inmediato de la izquierda respectivamente), el camino mínimo entonces es el camino mínimo hasta el único antecesor más el último movimiento. $\qed$

\subsection{Pseudocódigo}

% Explicacion del pseudocodigo, y de cómo se guarda el camino
% Representacion del diccionario

\subsubsection{Enfoque top-down vs. bottom-up}

% Dependencias, algoritmo iterativo...

\subsection{Complejidad del algoritmo}

\subsubsection{Complejidad en peor caso}

\subsubsection{Complejidad en mejor caso}

\subsection{Performance del algoritmo}


Como dijimos antes, la complejidad del algoritmo es siempre $\Theta(n m)$, sin distinción entre casos, por lo que el análisis de performance es simple.

\begin{figure}[H]
 \centering
	\includegraphics[width=0.9\textwidth]{img/exp/problema3-promedio.pdf}
	\caption{\footnotesize Tiempo que toma el algoritmo en $\mu$s para una entrada de tamaño $mn$.}
	\label{fig:problema3-promedio}
\end{figure}

En esta imagen se ve que se comporta como debe. Sin embargo, al igual que en los problemas anteriores, para confirmarlo totalmente, realizamos el gráfico de $\frac{T(nm)}{nm}$, dado que si esta función tiene a una constante cuando $nm \to \infty$, habremos confirmado experimentalmente que la complejidad del algoritmo es de $\Theta(n m)$.

\begin{figure}[H]
 \centering
	\includegraphics[width=0.9\textwidth]{img/exp/problema3-promedio2.pdf}
	\caption{\footnotesize Tiempo que toma el algoritmo en $\mu$s dividido $mn$ para una entrada de tamaño $mn$.}
	\label{fig:problema3-promedio2}
\end{figure}


\subsubsection{M\'etodo de experimentación}

Dados $n$ y $m$, generamos una matriz de $n \times m$, donde cada celda tiene un peso al azar. Para cada par $n$, $m$ generamos varias matrices (cada una con pesos distintos en cada celda), y tomamos la mediana de esas mediciones. 

De todas maneras, la varianza del tiempo para cada matriz de la misma dimensión era casi nula, dado que obviamente el valor de las celdas no afecta el tiempo. Sin embargo, nos parece importante aclarar que esto sucede, y que además fue verificado experimentalmente como dijimos.



