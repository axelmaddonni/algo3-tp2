\subsection{Explicación formal del problema}
Sea $G=(V,E)$ un grafo simple conexo con $n\geq 2$ vértices y $m$ aristas. Además sea $M\subseteq E$ tal que $M\neq \emptyset$. Se desea hallar un camino (no necesariamente simple) de $v_0$ a $v_{n-1}$ que pase al menos dos veces por un eje de $M$ (potencialmente el mismo) y que tenga longitud mínima.

En la figura (\ref{fig:ej-1.1}) pueden verse tres ejemplos. Los tres son muy parecidos pero permiten ilustrar distintas situaciones. En el primero (de izquierda a derecha y de arriba a abajo) encontramos el camino simple $P=(v_0,v_1,v_3,v_4,v_5)$ de longitud 4 que cumple con lo pedido pues tiene dos aristas especiales y es de longitud mínima. 

En el segundo se agregó una arista corriente entre los nodos $v_0$ y $v_5$. Puede notarse que en este caso el mejor camino es $P=(v_0,v_2,v_0,v_5)$ de longitud 3, el cual claramente no es simple pues pasa dos veces por el vértice 0. 

En el último caso, la arista recientemente agregada pasa a ser especial. Al hacer esto tenemos dos caminos de longitud mínima entre los que pasan por dos aristas especiales: el $P$ que encontramos antes, y $P'=(v_0, v_5, v_0, v_5)$. Notar que $P'$ no solo no es simple, sino que también usa a $v_5$ como nodo intermedio. Cualquiera de los dos es igualmente aceptable.

\begin{figure}[H]
\begin{minipage}{0.5\textwidth}
\centering
\begin{tikzpicture}[shorten >=1pt,auto,node distance=1.9cm,
                    semithick]
  \tikzstyle{every state}=[fill=red,draw=none,text=white]

	\node[state, inner sep=3pt,minimum size=0pt, fill=blue]	(0)		 						  {$v_0$};
	\node[state, inner sep=3pt,minimum size=0pt]	(1) [right of=0, above of=0]  {$v_1$};
	\node[state, inner sep=3pt,minimum size=0pt]	(2) [right of=0, below of=0]  {$v_2$};
	\node[state, inner sep=3pt,minimum size=0pt]	(3) [right of=1] 						  {$v_3$};
	\node[state, inner sep=3pt,minimum size=0pt]	(4) [right of=3] 						  {$v_4$};
	\node[state, inner sep=3pt,minimum size=0pt, fill=blue]	(5) [below of=4, right of=4]  {$v_5$};
	\path	(0) edge 	[ultra thick]				node {} (1)
           	edge  [red]								node {} (2)
				(1) edge  [red, ultra thick]  node {} (3)
						edge	[ultra thick]				node {} (0)        
				(2) edge  [red]								node {} (0)
        (3)	edge  [red, ultra thick] 	node {} (4)
        		edge	[red, ultra thick]	node {} (1)
        (4) edge	[ultra thick]				node {} (5)
        		edge	[red, ultra thick]	node {} (3)
        (5) edge	[ultra thick]				node {} (4);

\end{tikzpicture}
\end{minipage}%
\begin{minipage}{0.5\textwidth}
\centering
\begin{tikzpicture}[shorten >=1pt,auto,node distance=1.9cm,
                    semithick]
  \tikzstyle{every state}=[fill=red,draw=none,text=white]

	\node[state, inner sep=3pt,minimum size=0pt, fill=blue]	(0)		 						  {$v_0$};
	\node[state, inner sep=3pt,minimum size=0pt]	(1) [right of=0, above of=0]  {$v_1$};
	\node[state, inner sep=3pt,minimum size=0pt]	(2) [right of=0, below of=0]  {$v_2$};
	\node[state, inner sep=3pt,minimum size=0pt]	(3) [right of=1] 						  {$v_3$};
	\node[state, inner sep=3pt,minimum size=0pt]	(4) [right of=3] 						  {$v_4$};
	\node[state, inner sep=3pt,minimum size=0pt, fill=blue]	(5) [below of=4, right of=4]  {$v_5$};
	\path	(0) edge 											node {} (1)
           	edge  [red, ultra thick]	node {} (2)
						edge 	[ultra thick]				node {} (5)
				(1) edge  [red]							  node {} (3)
						edge											node {} (0)        
				(2) edge  [red, ultra thick]	node {} (0)
        (3)	edge  [red]							 	node {} (4)
        		edge	[red]								node {} (1)
        (4) edge											node {} (5)
        		edge	[red]	node {} (3)
        (5) edge											node {} (4)
        		edge	[ultra thick]				node {} (0);

\end{tikzpicture}
\end{minipage}%

\vspace{15pt}
\centering
\begin{minipage}{0.5\textwidth}
\begin{tikzpicture}[shorten >=1pt,auto,node distance=2cm,
                    semithick]
  \tikzstyle{every state}=[fill=red,draw=none,text=white]

	\node[state, inner sep=3pt,minimum size=0pt, fill=blue]	(0)		 						  {$v_0$};
	\node[state, inner sep=3pt,minimum size=0pt]	(1) [right of=0, above of=0]  {$v_1$};
	\node[state, inner sep=3pt,minimum size=0pt]	(2) [right of=0, below of=0]  {$v_2$};
	\node[state, inner sep=3pt,minimum size=0pt]	(3) [right of=1] 						  {$v_3$};
	\node[state, inner sep=3pt,minimum size=0pt]	(4) [right of=3] 						  {$v_4$};
	\node[state, inner sep=3pt,minimum size=0pt, fill=blue]	(5) [below of=4, right of=4]  {$v_5$};
	\path	(0) edge 											node {} (1)
           	edge  [red]								node {} (2)
						edge 	[red, ultra thick]	node {} (5)
				(1) edge  [red]							  node {} (3)
						edge											node {} (0)        
				(2) edge  [red]								node {} (0)
        (3)	edge  [red]							 	node {} (4)
        		edge	[red]								node {} (1)
        (4) edge											node {} (5)
        		edge	[red]	node {} (3)
        (5) edge											node {} (4)
        		edge	[red, ultra thick]				node {} (0);

\end{tikzpicture}
\end{minipage}%
\caption{\footnotesize{Ejemplos del problema. Las aristas especiales están pintadas de rojo y las comunes de negro. Las aristas que pertenecen a la solución están engrosadas. En azul distinguimos a los nodos inicial y final.}}
  \label{fig:ej-1.1}
\end{figure}

\subsection{Explicación de la solución}
Si bien lo que se pide es, en definitiva, encontrar un camino mínimo en $G$, la dificultad adicional que implica hacer que el camino contenga al menos dos aristas de $M$ hace que no podamos aplicar de forma directa los algoritmos clásicos para este propósito. Para solventar esto consideraremos un grafo alternativo a $G$, $G'$, con el cual resolver el problema planteado originalmente será equivalente a encontrar un camino mínimo en $G'$ de forma tradicional (en este caso utilizando BFS).

\subsubsection{Nuevo grafo para el modelado}
Lo primero que haremos es armarnos un grafo nuevo $G'$ a partir de $G$. 

Consideramos tres grafos isomorfos a $G$: $G_0=(V_0,E_0)$, $G_1=(V_1,E_1)$ y $G_2=(V_2,E_2)$, tales que $f_k:V\rightarrow V_k/ f_k(v_i) = v_{k\times n + i}$ para $k\in \{0,1,2\}$ son las biyecciones correspondientes. En particular, $G_1 = G$. La figura (\ref{fig:ej-1.2}) ilustra la situación para un ejemplo puntual.

\begin{figure}[H]
\begin{minipage}{0.33\textwidth}
\centering
\begin{tikzpicture}[shorten >=1pt,auto,node distance=1.9cm,
                    semithick]
  \tikzstyle{every state}=[fill=red,draw=none,text=white]

	\node[state, inner sep=3pt,minimum size=0pt, fill=blue]	(0)		 						  {$v_0$};
	\node[state, inner sep=3pt,minimum size=0pt]	(1) [left of=0, below of=0]  {$v_1$};
	\node[state, inner sep=3pt,minimum size=0pt]	(2) [right of=0, below of=0]  {$v_2$};
	\node[state, inner sep=3pt,minimum size=0pt]	(3) [below of=1] 						  {$v_3$};
	\node[state, inner sep=3pt,minimum size=0pt]	(4) [below of=2] 						  {$v_4$};
	\node[state, inner sep=3pt,minimum size=0pt, fill=blue]	(5) [below of=4, left of=4]  {$v_5$};
	\path	(0) edge 											node {} (1)
           	edge  [red]								node {} (2)
				(1) edge  [red]							  node {} (3)
						edge											node {} (0)        
				(2) edge  [red]								node {} (0)
						edge											node {} (4)
        (3)	edge	[red]								node {} (1)
        		edge											node {} (5)
        (4) edge	[red]								node {} (5)
        		edge											node {} (2)
        (5) edge	[red]								node {} (4)
        		edge											node {} (3);

\end{tikzpicture}
\end{minipage}%
\begin{minipage}{0.33\textwidth}
\centering
\begin{tikzpicture}[shorten >=1pt,auto,node distance=1.9cm,
                    semithick]
  \tikzstyle{every state}=[fill=red,draw=none,text=white]

	\node[state, inner sep=3pt,minimum size=0pt, fill=blue]	(0)		 						  {$v_6$};
	\node[state, inner sep=3pt,minimum size=0pt]	(1) [left of=0, below of=0]  {$v_7$};
	\node[state, inner sep=3pt,minimum size=0pt]	(2) [right of=0, below of=0]  {$v_8$};
	\node[state, inner sep=3pt,minimum size=0pt]	(3) [below of=1] 						  {$v_9$};
	\node[state, inner sep=3pt,minimum size=0pt]	(4) [below of=2] 						  {$v_{10}$};
	\node[state, inner sep=3pt,minimum size=0pt, fill=blue]	(5) [below of=4, left of=4]  {$v_{11}$};
	\path	(0) edge 											node {} (1)
           	edge  [red]								node {} (2)
				(1) edge  [red]							  node {} (3)
						edge											node {} (0)        
				(2) edge  [red]								node {} (0)
						edge											node {} (4)
        (3)	edge	[red]								node {} (1)
        		edge											node {} (5)
        (4) edge	[red]								node {} (5)
        		edge											node {} (2)
        (5) edge	[red]								node {} (4)
        		edge											node {} (3);

\end{tikzpicture}
\end{minipage}%
\vspace{15pt}
\centering
\begin{minipage}{0.33\textwidth}
\begin{tikzpicture}[shorten >=1pt,auto,node distance=1.9cm,
                    semithick]
  \tikzstyle{every state}=[fill=red,draw=none,text=white]

	\node[state, inner sep=3pt,minimum size=0pt, fill=blue]	(0)		 						  {$v_{12}$};
	\node[state, inner sep=3pt,minimum size=0pt]	(1) [left of=0, below of=0]  {$v_{13}$};
	\node[state, inner sep=3pt,minimum size=0pt]	(2) [right of=0, below of=0]  {$v_{14}$};
	\node[state, inner sep=3pt,minimum size=0pt]	(3) [below of=1] 						  {$v_{15}$};
	\node[state, inner sep=3pt,minimum size=0pt]	(4) [below of=2] 						  {$v_{16}$};
	\node[state, inner sep=3pt,minimum size=0pt, fill=blue]	(5) [below of=4, left of=4]  {$v_{17}$};
	\path	(0) edge 											node {} (1)
           	edge  [red]								node {} (2)
				(1) edge  [red]							  node {} (3)
						edge											node {} (0)        
				(2) edge  [red]								node {} (0)
						edge											node {} (4)
        (3)	edge	[red]								node {} (1)
        		edge											node {} (5)
        (4) edge	[red]								node {} (5)
        		edge											node {} (2)
        (5) edge	[red]								node {} (4)
        		edge											node {} (3);

\end{tikzpicture}
\end{minipage}%
\caption{\footnotesize{Tres isomorfismos del grafo original, construidos de la forma indicada.}}
  \label{fig:ej-1.2}
\end{figure}


Además, definimos $M'$, un conjunto de arcos (aristas orientadas) de peso 1, como 

{\footnotesize$M'= \{(f_0(v), f_1(w))$ y $(f_0(w), f_1(v)) / (v,w)\in M\} \cup \{(f_1(v), f_2(w)) $ y $(f_1(w), f_2(v))/ (v,w)\in M\}$}

Por ejemplo, en la figura (\ref{fig:ej-1.2}) la arista $(v_0, v_2)\in M$. Entonces queremos que $M'$ tenga los arcos $(v_0, v_8)$, $(v_2, v_6)$, $(v_6, v_{14})$ y $(v_8, v_{12})$.
Observar que en $M'$ solo hay arcos que van de nodos de $G_0$ a nodos de $G_1$, y de $G_1$ a $G_2$. En ningún caso hay arcos de $G_t$ a $G_h$, con $h<t$; ni arcos que vayan directamente de $G_0$ a $G_2$.

Entonces hasta acá tenemos tres grafos conexos isomorfos. Podemos pensarlos como las tres componentes conexas de un grafo con $3n$ vértices y $3m$ aristas. A continuación uniremos estas tres componentes mediante los arcos de $M'$: sea $G^*=(V_1\cup V_2\cup V_3, E_1\cup E_2\cup E_3 \cup M')$. 

Podemos pensar a $G^*$ con la siguiente analogía: cada una de las tres componentes es un nivel, y los arcos ``escaleras mecánicas'' que permiten subir de un nivel al siguiente en forma unidireccional y que no se saltea niveles. Notar que entonces $G'$ no es fuertemente conexo pues desde un nodo del nivel 2 o 3 no puedo alcanzar a un nodo en el nivel 1. Sin embargo, para cualquier vértice en el nivel 1 todos los vértices del grafo son alcanzables. En particular esto vale para $v_0$.  

Finalmente, definimos a $G'$ como el resultado de quitarle a los niveles 0 y 1 de $G^*$ todas las ``aristas isomorfas'' a las aristas de $M$\footnote{En rigor, tanto $G'$ como $G^*$ sirven a nuestro propósito, pero $G'$ tiene la ventaja de que permitirá reducir el uso de memoria (potencialmente mucho si hay muchas aristas especiales) y constantes en la complejidad de la implementación.}. En este caso ya no es cierto que cualquier vértice de $G'$ sea alcanzable desde cualquier vértice del primer nivel: en efecto, si vemos la figura (\ref{fig:ej-1.2}), al quitarle las aristas especiales al primer nivel el nodo $v_5$ no es alcanzable desde el $v_1$. No obstante, sigue valiendo el siguiente lema:

\paragraph*{Lema 1.1: }Todo vértice del tercer nivel (nivel 2) de $G'$ es alcanzable desde todo vértice del primer nivel (nivel 0). 

\paragraph*{Demostración: }Sean $u,w\in V_1$ (es decir, ambos del nivel 0). Supongamos que $w$ no es alcanzable desde $u$ en $G'$. Como $G_1$ inicialmente era conexo, esto significa que existía camino de $u$ a $w$, $Q$, y seguro incluía alguna arista especial (sino seguiría existiendo en $G'$ y $w$ sería alcanzable desde $u$). Como todos los nodos que eran incidentes a una arista especial en $G_1$ son incidentes a un arco en $G'$, seguro existe $z\in Q$, tal que $z$ es incidente a un arco que permite pasar al segundo nivel. Luego, es posible llegar desde $u$ a algún vértice del segundo nivel. Si no existe tal nodo $w$, entonces cualquier nodo del primer nivel es alcanzable desde $u$, y como $M$ no era vacío, entonces seguro hay un camino desde $u$ hasta el segundo nivel.

Es fácil ver que exactamente el mismo razonamiento se puede realizar para probar que es posible llegar desde cualquier nodo del segundo nivel a algún nodo del tercer nivel. Luego, por concatenación de ambas cosas, es posible llegar desde cualquier nodo del primer nivel a algún nodo del tercero. Pero como el tercer nivel sigue siendo conexo (pues nunca quitamos las aristas especiales) entonces es claro que esto es equivalente a poder llegar a cualquier nodo del tercer nivel. $\qed$

Debido a este lema, y como todas las aristas de $G'$ tienen peso 1, es posible aplicar el algoritmo BFS para hallar el camino mínimo entre un nodo del nivel 0 y otro del 2. Particularmente, entre los nodos 0 y $3n-1$. 

En la sección siguiente probaremos que esto es equivalente a resolver el problema planteado originalmente.

\subsubsection{Correctitud y optimalidad}
La siguiente proposición garantiza la correctitud de nuestra solución.

\paragraph*{Proposición 1.1: }
Sea $P'=(v_0, u_1,\dots, u_{k}, v_{3n-1})$ un camino mínimo de $v_0$ a $v_{3n-1}$ en $G'$, de longitud $k+1$, entonces $P = (P'$ mod $n$) \footnote{$P = P'$ mod $n \Leftrightarrow P_i = v_{j\text{ mod }n}\text{ donde $P'_i=v_j$ }, i = 0,\dots, k-1$}es camino de $v_0$ a $v_{n-1}$ en G, mínimo entre los que pasan por al menos dos aristas especiales.

\paragraph*{Demostración: }

Hay que ver tres cosas respecto de $P$: que es camino de $v_0$ a $v_{n-1}$, que pasa por al menos dos aristas especiales, y que es mínimo respecto a los caminos que cumplen ambas cosas.

Va a ser útil recordar que $f_k:V\rightarrow V_k/ f_k(v_i) = v_{k\times n + i}$ para $k\in \{0,1,2\}$ son las biyecciones de los isomorfismos planteados en la sección anterior.

\begin{itemize}
	\item Es camino: Ante todo, por definición del operador módulo vale que $(\forall v_i\in P)$ $0\leq i$ mod $n\leq n-1$. Es decir que todos los vértices de $P$ pertenecen a $G$, y además empieza en $v_0$ y termina en $v_{n-1} = v_{(3n-1)\text{ mod } n}=v_{(2n+n-1)\text{ mod } n} $. Queda ver que efectivamente nodos consecutivos en $P$ son adyacentes en $G$. \\Si $v_i, v_j\in P$ son consecutivos entonces $v_{h+i}\in P'$ tiene que ser adyacente con $v_{h'+j}\in P'$, donde $h$ y $h'$ son un par de constantes múltiplos de $n$. Por el contexto del problema $h$ y $h'$ solo pueden ser $0, n$ o $2n$. Luego, si $h=h'=k\times n$, por definición de isomorfismo y por cómo está definida $f_k$, si $v_{h+i}$ es adyacente a $v_{h+j}$ en $G_k$ (cosa que pasa, sino no podría pasar en $G'$) entonces $v_i$ es adyacente a $v_j$ en $G$. Si $h\neq h'$, seguro que cada nodo es el extremo de un arco (pues están en diferentes niveles). Pero por construcción de $G'$, solo puede haber un arco entre $v_{h+i}$ y $v_{h'+j}$ si había una arista especial entre $v_i$ y $v_j$ en $G$, lo que significa que eran adyacentes. Luego, queda probado que $P$ es un camino válido de $v_0$ a $v_{n-1}$.
	\item Pasa por al menos dos aristas especiales: el nodo $v_{3n-1}$ pertenece al nivel 2 de $G'$. Esto significa que paso por dos arcos. Un arco entre $v_{k\times n + i}$ y $v_{(k+1)\times n + j}$ en $G'$ solo existe si $(v_i, v_j)\in M$. Por definición de $P$, si tal arco pertenece a $P'$ entonces tal arista especial pertenece a $P$. Luego, $P$ tiene al menos dos aristas de $M$ (podría tener más, pues en el tercer nivel las aristas especiales siguen existiendo y no son arcos).
	\item Es mínimo: Supongamos que $P$ no es óptimo para el problema. Entonces existe $Q$ tal que $|Q| < |P|$ y cumple con pasar por dos aristas de $M$. Construyamos $Q'$, un camino de $v_0$ a $v_{3n-1}$ en $G'$, basado en $Q$.\\
	 La idea es la siguiente: recorremos las aristas de $Q$ en orden y las vamos poniendo en $Q'$ hasta encontrar la primer arista especial, $(v_i, v_j)$. En su lugar agregamos el arco $(v_i, v_{n+j})$ a $Q'$. Seguimos completando $Q'$ con aristas $(f_1(u), f_1(w))$ por cada arista común $(u,w)$ que encontramos en $Q$. Eventualmente llegamos a una segunda arista especial (por hipótesis existe), $(v_s, v_t)$, y en su lugar agregamos el arco $(v_{n+s}, v_{2n+t})$ a $Q'$. Ahora bien, en este punto si $Q$ es mínimo lo mejor que puede hacer es tomar el camino de distancia mínima desde $v_t$ hasta $v_{n-1}$. Pero tal camino es isomorfo a un camino desde $v_{2n+t}$ hasta $v_{3n-1}$, pues el tercer nivel es isomorfo a $G$. Por lo tanto agregando este camino a $Q'$, llegamos a que $Q'$ es un camino de $v_0$ a $v_{3n-1}$ en $G'$. \\
	 Pero como $|Q'|=|Q|<|P|=|P'|$, llegamos a que hay un camino más corto que $P'$ en $G'$ que conecta los mismos vértices. Esto es absurdo, pues por hipótesis $P'$ era camino mínimo. Luego, el absurdo provino de suponer que $P$ no era óptimo para el problema.
\end{itemize}

Finalmente, queda demostrada la proposición. $\qed$

De hecho, la recíproca de esta proposición también vale. No lo probamos sin embargo porque no es necesario para la correctitud de la solución. La forma de probarlo sería similar igualmente.

\subsubsection{Explicación del código}

Notar que para las dimensiones del grafo que toma BFS no usamos $n$ y $m$ sino $n'$ y $m'$, para no confundir con las dimensiones del problema original porque pueden ser diferentes, y de hecho por cómo lo vamos a usar, así va a ser.

\begin{algorithm}[H]
  \begin{algorithmic}[1]
  \caption{Pseudocódigo del procedimiento BFS}
  \label{algo:1-1}
    \Procedure{bfs}{\texttt{ListaAdyacencia} $vs$, \texttt{vertice} $root$, \texttt{vertice} $target$, \texttt{int} $n'$}$\rightarrow$ \texttt{Vector<vertice>}
    	\State \texttt{cola<vertice>} $c \gets Vacia()$
    	\Comment $O(1)$
    	\State \texttt{vector<int>} $distancia(n', \infty)$
    	\Comment $O(n')$
    	\State \texttt{vector<vertice>} $acm(n', -1)$
    	\Comment $O(n')$
    	\State $distancia[root]\gets 0$
    	\Comment $O(1)$
    	\State $acm[root]\gets root$
    	\Comment $O(1)$
    	\State $c.push(root)$
    	\Comment $O(1)$
    	\While{$\neg c.vacia?()$}
    		\Comment $O(n)$ veces
    		\State $actual\gets c.pop()$
    		\Comment $O(1)$
    		\State \texttt{VerticesAdyacentes} $vecinos\gets vs[actual]$
    		\Comment $O(1)$
    		\For{$v \in vecinos$}
    		\Comment $O(d(v))$ veces
    			\If{$distancia[v] = \infty$}
    			\Comment $O(1)$
    				\State $distancia[v]\gets distancia[actual] + 1$
    				\Comment $O(1)$
    				\State $acm[v] = actual$
    				\Comment $O(1)$
    				\If{$v=target$}
    				\Comment $O(1)$
    					\State $break$
    					\Comment $O(1)$
    				\EndIf
    				\State $c.push(v)$
    				\Comment $O(1)$
    			\EndIf
    		\EndFor
    	\EndWhile
    	\State \texttt{int} $long\_sol\gets distancia[target]-1$
    	\Comment $O(1)$
    	\State \texttt{vector<vertice>} $solucion(long\_sol, 0)$
    	\Comment $O(long\_sol)\subseteq O(m')$
    	\State \texttt{vertice} $v\gets acm[target]$
    	\Comment $O(1)$
    	\For{\texttt{int} $i$ desde $long\_sol - 1$ hasta $0$}
    	\Comment $O(m')$ veces
    		\State $solucion[i]\gets v$
    		\Comment $O(1)$
    		\State $v\gets acm[v]$
    		\Comment $O(1)$
    	\EndFor
    	\State \texttt{return} $solucion$
		\EndProcedure
	\end{algorithmic}
\end{algorithm}

La implementación de BFS que realizamos está compuesta por dos partes: la primera hasta la línea 17 inclusive, es la implementación clásica del algoritmo de búsqueda en anchura para determinar caminos mínimos, en particular modificada un poco para que termine a penas compute un camino hasta el nodo objetivo pues es el único que nos importa realmente; la segunda consiste en reconstruir el camino mínimo entre $root$ y $target$ a partir del árbol de caminos mínimos. Notar que la función tiene como precondición que efectivamente exista algún camino desde $root$ hasta $target$.

Para la primer parte tenemos esencialmente tres estructuras importantes: 
\begin{itemize}
\item una cola FIFO de vértices donde iremos encolando los vecinos del nodo en el que estamos actualmente y que todavía no hayamos visitado; la misma está implementada sobre una lista doblemente enlazada, lo que permite que las operaciones de encolar, desencolar y ver el siguiente elemento sean todas $O(1)$.
\item un vector de distancias tal que la posición i-ésima del mismo guarda la distancia desde el $root$ hasta el nodo $i$ (o bien $\infty$ si todavía no pasamos por $i$, o simplemente $i$ no es alcanzable desde $root$).
\item un vector de vértices que representará nuestro árbol de caminos mínimos desde el $root$ hasta cualquier nodo, de forma que en la posición i-ésima del vector tendremos al padre del nodo $i$ en el árbol (o bien -1, si todavía no pasamos por $i$, o simplemente no es alcanzable desde $root$).
\end{itemize}

Que la búsqueda es correcta es resultado inmediato de que el algoritmo es el BFS tradicional cuya correctitud ya está probada.

Una vez hallado un camino mínimo hasta el nodo $target$, queremos ahora armar un vector de vértices que contenga a todos los vértices de dicho camino. Como la distancia es la cantidad de vértices en el camino menos uno, entonces el vector tendrá que tener tamaño igual a la distancia más uno. Pero como no nos interesa que el primer y último nodos estén en el vector entonces nos queda que el largo del mismo será $l=d(root, target) - 1$. Luego, es cuestión de llenar las posiciones de este vector de atrás para adelante, pues \emph{a priori} para cada nodo solo sabemos cual es su padre en el árbol de caminos mínimos. La última posición tendrá al padre del nodo $target$, la anteúltima al padre del padre y así. Iterando $l$ veces llegamos a que en la posición inicial del vector hay un nodo que es hijo de $root$ y ancestro de $target$.

Finalmente devolvemos este vector.

\begin{algorithm}[H]
  \begin{algorithmic}[1]
  \caption{Pseudocódigo del main}
  \label{algo:1-2}
    \Procedure{main}{}
    	\State \texttt{int} $n$
    	\Comment {Cantidad de nodos}
    	\State \texttt{vector(vector(int))} $input$
    	\Comment {\small $input[i]$ almacena los datos de la i-ésima arista pasada}\normalsize
    	\State $inicializar(input, n)$
    	\Comment \small Leemos los datos pasados como parámetros e inicializamos \normalsize
    	\State \texttt{ListaAdyacencia} $adj\_list(3*n, VerticesAdyacentes()$)
    	\Comment $O(3\times n) = O(n)$
    	\For{$(v_1, v_2, e) \in input$}
    	\Comment $m$ veces
    		\If{$e = True$}
    		\Comment $O(1)$
    			\State $adj\_list[v_1].push\_back(v_2 + n)$
    			\Comment $O(1)$
    			\State $adj\_list[v_1 + n].push\_back(b + 2\times n)$
    			\Comment $O(1)$
    			\State $adj\_list[v_2].push\_back(v_1 + n)$
    			\Comment $O(1)$
    			\State $adj\_list[v_2 + n].push\_back(v_1 + 2\times n)$
    			\Comment $O(1)$
    		\Else
     			\State $adj\_list[v_1].push\_back(v_2)$
     			\Comment $O(1)$
     			\State $adj\_list[v_2].push\_back(v_1)$
     			\Comment $O(1)$
     			\State $adj\_list[v_1 + n].push\_back(v_2 + n)$
     			\Comment $O(1)$
     			\State $adj\_list[v_2 + n].push\_back(v_1 + n)$
     			\Comment $O(1)$
    		\EndIf
    		\State $adj\_list[v_1 + 2\times n].push\_back(v_2 + 2\times n)$
    		\Comment $O(1)$
    		\State $adj\_list[v_2 + 2\times n].push\_back(v_1 + 2\times n)$
    		\Comment $O(1)$
    	\EndFor
    	\State \texttt{vector<vertice>} $solucion\gets bfs(adj\_list, 0, 3\times n)$
    	\Comment $O(3n + 3m) = O(n + m)$
    	\State $print(solucion.size() + 1)$
    	\For{$v \in solucion$}
    		\State $print(v \% n)$
    	\EndFor
		\EndProcedure
  \end{algorithmic}
  \end{algorithm}

Nuestra función main tiene tres partes importantes:
\begin{itemize}
	\item El armado de la lista de adyacencias de $G'$.
	\item El llamado a BFS pasando como parámetros la lista de adyacencias anterior, tomando como $root$ el nodo $0$ y como target el $3n-1$. Dichos parámetros cumplen las precondiciones de BFS por el Lema 1.1 y por ser todas aristas de peso 1.
	\item La impresión del resultado. Acá es importantísimo notar que imprimimos los vértices módulo $n$ pues lo que nos devuelve BFS son nodos del grafo $G'$ y no de $G$. Por la Proposición 1.1 esto efectivamente constituye una solución al problema original.
\end{itemize}

\subsection{Complejidad del algoritmo}
La complejidad en peor caso de la solución es la complejidad de la función $main$. Omitiendo las partes de lectura y escritura de datos, tenemos que el costo de dicha función es el costo de armar el nuevo grafo más el costo de realizar $BFS$ sobre él.

Viendo el algoritmo \ref{algo:1-2}, el costo de armar el grafo es $O(n+6m) = O(n+m)$. Vale destacar que esta complejidad es además claramente una cota inferior, pues el costo de armar el grafo nuevo depende únicamente de la cantidad de vértices y aristas, y no de las características topológicas particulares. Por lo tanto el algoritmo en general debe ser al menos $\Omega(n+m)$.

Por otra parte, observando el algoritmo \ref{algo:1-1}, $BFS$ tiene una complejidad en peor caso de 

\begin{equation}
\begin{aligned}
	O(1+2n+3+2n+(\sum_{i=0}^{n-1}d(v_{i}))\times 5 + m + 1 + 2m) & = O(5+4n+2m\times 5 + 3m) \\
	&= O(4n+13m) \\
	&= O(n+m)
\end{aligned}
\end{equation}

Notar que en el primer término podemos escribir la sumatoria de los grados de todos los nodos debido a que en peor caso hará falta pasar por todos ellos, y por otra parte sabemos que pasamos por cada uno exactamente una vez. El segundo término resulta de agrupar y reemplazar la sumatoria por $2m$ (cosa que podemos hacer pues es una identidad válida para todos los grafos).

Luego, la complejidad asintótica del algoritmo en peor caso es $O(n+m+3n+3m) = O(4(n+m)) = O(n+m)$. Por otra parte como dijimos que también era $\Omega(n+m)$, tenemos que es $\Theta(n+m)$.

De hecho, asintóticamente también lo es en mejor caso (cuando existe un camino de longitud 3): si bien $BFS$ puede ser $\Theta(1)$ debido a que nuestra implementación termina de buscar una vez que encuentra al nodo deseado, armar el grafo sigue siendo $\Theta(n+m)$ en cualquier caso.

\subsection{Performance del algoritmo}

\subsubsection{M\'etodo de experimentación}